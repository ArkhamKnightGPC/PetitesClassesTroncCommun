\documentclass[french]{article}
\usepackage[T1]{fontenc}
\usepackage[utf8]{inputenc}
\usepackage{lmodern}
\usepackage[a4paper]{geometry}
\usepackage{babel}
\usepackage{amsmath}
\usepackage{amsfonts}
\usepackage{tcolorbox}
\usepackage{color}
\usepackage{breqn}

\begin{document}
	\title{DM1: Une preuve du théorème d'Ascoli}
	\date{Dernière modification \today}
	
	\maketitle
	
	\begin{tcolorbox}[colback=gray!5!white,colframe=gray!75!black]
		\textbf{\large{Q1}} Montrer que si $(Y,d)$ est un espace métrique compact et $(E,||\cdot||)$ un espace de Banach, alors, $\mathcal{C}(Y;E)$ l'espace des fonctions continues de Y dans E, muni de la norme de la convergence uniforme $||f||_{\infty} = \sup_{x \in Y} ||f(x)||$, est un espace de Banach.
	\end{tcolorbox}

	Soit $(f_n)_{n \geq 0}$ une suite de Cauchy d'éléments de $\mathcal{C}(Y;E)$. On veut montrer que $(f_n)_{n \geq 0}$ est convergente.
	
	Par définition, \[ \forall \epsilon > 0 \quad \exists n_0 \in \mathbb{N} \quad \text{tel que} \quad  \forall n,m \geq n_0 \quad ||f_n - f_m||_{\infty} < \epsilon.\]
	
	Donc,	
	\begin{dmath*}
		\forall x \in Y \quad ||f_n(x) - f_m(x)|| < \epsilon \\ \iff  \text{$(f_n(x))_{n \geq 0}$ est une suite de Cauchy dans $(E, ||\cdot||)$ qui est un espace métrique complet} \\ \iff \text{$(f_n(x))_{n \geq 0}$ converge vers une limite $z_x \in E$}.
	\end{dmath*}
	
	Soit $z : Y \to E$ tel que $\forall x \in Y \quad z(x) = z_x$.
	
	On note que \[\forall x \in Y \lim_{m \to +\infty} ||f_n(x) - f_m(x)|| = ||f_n(x) - z(x)|| < \epsilon.\]
	
	Ce qui implique $||f_n - z||_{\infty} < \epsilon  \quad \forall n \geq n_0$. Donc, $(f_n)_{n \geq 0}$ converge uniformemet vers $z$.
	
	On note que $z \in \mathcal{C}(Y;E)$, parce que $(f_n)_{n \geq 0}$ est continue.
	
	\begin{tcolorbox}[colback=green!5!white,colframe=green!75!black]
		\textbf{Remarque} On note que la convergence uniforme est essentiel pour concluire. Par exemple, soit 
		\[f_n(x) = \arctan(nx).\]
		La limite de cette suite est 
		\begin{equation}
			z(x) = 
			\begin{cases}
				-\frac{\pi}{2} \quad \text{si $x < 0$}\\
				+\frac{\pi}{2} \quad \text{si $x > 0$},
			\end{cases}
		\end{equation}
		qui est donc discontinue en $x = 0$.
	\end{tcolorbox}
	 
	 \newpage
	 
	 \begin{tcolorbox}[colback=yellow!5!white,colframe=yellow!75!black]
	 	\textbf{\large{Définition}} Un espace métrique $(X,d)$ est dit \textbf{précompact} si pour tout $\epsilon > 0$, il existe des élements $x_1, ..., x_N$ de $X$ tels que $X \subset \cup_{i=1}^{N} B(x_i, \epsilon)$.
	 \end{tcolorbox}
 
 	\begin{tcolorbox}[colback=gray!5!white,colframe=gray!75!black]
 		\textbf{\large{Q2}} Montrer que $[0,1]$ est précompact dans $\mathbb{R}$.
 	\end{tcolorbox}
 
	$\forall \epsilon > 0$ on veut montrer qu'il existe des éléments $ x_0, x_1, ..., x_N \in [0,1] \quad N \in \mathbb{N} \text{  tels que  } [0,1] \subset \cup_{i=0}^{N} B(x_i, \epsilon)$.
	
	Pour $\epsilon > \frac{1}{2}$ on note que $X \subset B(\frac{1}{2}, \epsilon)$.
	
	Supposons $\epsilon \leq \frac{1}{2}$. Soit $N = \lfloor \frac{1}{\epsilon} \rfloor$, posons $\forall i, x_i = i\epsilon$.
	
	On note que $\forall x \in [0, 1], x \in B( x_{\lfloor \frac{x}{\epsilon} \rfloor},\epsilon)$. Donc, $[0,1] \subset \cup_{i=0}^{N} B(x_i, \epsilon)$. 

 	\begin{tcolorbox}[colback=gray!5!white,colframe=gray!75!black]
 		\textbf{\large{Q3}} Montrer que tout sous-ensemble borné de $\mathbb{R}^N$ est précompact.
 	\end{tcolorbox}
 
 	Soit $A$ un sous-ensemble borné de $\mathbb{R}^N$. Supposons, par absurde, que $A$ n'est pas précompact.
 	
 	Donc, il existe $r>0$ tel que il n'existe pas de recouvrement fini de $A$ par des boules ouvertes de rayon $r$
 	
 	$\implies $ cette recouvrement ne peut pas être dans une boule de rayon fini
 	
 	$\implies A$ n'est pas borné.
	
 	\begin{tcolorbox}[colback=gray!5!white,colframe=gray!75!black]
 		\textbf{\large{Q4}} Montrer qu'un espace compact est précompact.
 	\end{tcolorbox}
 
 	Soit $(X,d)$ un espace métrique compact. Supposons, par absurde, que $(X,d)$ n'est pas précompact.
 	
 	Donc, il existe $r>0$ tel que il n'existe pas de recouvrement fini de $X$ par des boules ouvertes de rayon $r$.
 	
 	Donc, par récurrence, on peut construire une suite $(x_n)_{n \geq 0}$ tel que $x_m \notin B(x_n, r) \implies d(x_n, x_m) \geq r \quad \forall n \neq m$. On note que $(x_n)_{n \geq 0}$ n'a pas de valeur d'adhérence. Donc, $X$ n'est pas compact.
 
 	\begin{tcolorbox}[colback=gray!5!white,colframe=gray!75!black]
 		\textbf{\large{Q5}} Soit $(X,d)$ un espace métrique et $Z \subset X$ précompact. Montrer que $\bar{Z}$ l'adhérence de $Z$ est précompacte.
 	\end{tcolorbox}
 
 	Soit $\epsilon > 0$, on veut construire une recouvrement fini de $\bar{Z}$ par des boules ouvertes de rayon $\epsilon$.
 	
 	$Z$ précompact $\implies \exists (x_1,...x_n) \in Z^n$ tel que $Z = \cup_{i=1}^{n} B(x_i, \frac{\epsilon}{2})$.
 	
 	$\bar{Z}$ est l'ensemble des limites de suites convergentes d'éléments de $Z$. Donc, $\forall x \in \bar{Z} \quad \exists y \in Z$ tel que $d(x, y) < \frac{\epsilon}{2}$. On note que $\exists 1 \leq i \leq n$ tel que $y \in B(x_i, \frac{\epsilon}{2})$.
 	
 	Par l'inegalité triangulaire,
 	
 	\[d(x, x_i) \leq d(x, y) + d(y, x_i) < \frac{\epsilon}{2} + \frac{\epsilon}{2} = \epsilon.\]
 	
 	Donc, $\bar{Z} = \cup_{i=1}^{n} B(x_i, \epsilon)$. 
 	
	\begin{tcolorbox}[colback=gray!5!white,colframe=gray!75!black]
		\textbf{\large{Q6}} Montrer que $(X,d)$ est compact si et seulement s'il est complet et précompact.
	\end{tcolorbox}

	Supposons $(X,d)$ compact. Soit $(x_n)_{n \geq 0}$ une suite de Cauchy dans $(X,d)$. $(X,d)$ compact implique que  $(x_n)_{n \geq 0}$ admet un valeur d'adhérence, donc $(x_n)_{n \geq 0}$ est convergente et $(X,d)$ complet.
	
	Réciproquement, supposons $(X,d)$ complet et précompact.
	
	Soit $(x_n)_{n \geq 0}$ une suite dans $(X,d)$. On veut montrer que $(x_n)_{n \geq 0}$ a un valeur d'adhérence.
	
	$(X,d)$ précompact $\implies$ par le Principe de Dirichlet que $\forall \epsilon > 0$ il existe une sous-suite $(x_{\phi(n)})_{n \geq 0}$ dans une boule ouvert de rayon $\epsilon$. Donc, $\forall n,m \quad d(x_n, x_m) < \epsilon$. Donc, $(x_{\phi(n)})_{n \geq 0}$ est une suite de Cauchy. $(X,d)$ complet implique que $(x_{\phi(n)})_{n \geq 0}$ converge $\implies \quad (x_n)_{n \geq 0}$ a un valeur d'adhérence.

	\newpage

	\begin{tcolorbox}[colback=gray!5!white,colframe=gray!75!black]
		\textbf{\large{Q7}} 
	\end{tcolorbox}
	
	\begin{tcolorbox}[colback=gray!5!white,colframe=gray!75!black]
		\textbf{\large{Q8}} 
	\end{tcolorbox}

	\begin{tcolorbox}[colback=gray!5!white,colframe=gray!75!black]
		\textbf{\large{Q9}} 
	\end{tcolorbox}

	\begin{tcolorbox}[colback=gray!5!white,colframe=gray!75!black]
		\textbf{\large{Q10}} 
	\end{tcolorbox}

	\begin{tcolorbox}[colback=gray!5!white,colframe=gray!75!black]
		\textbf{\large{Q11}} 
	\end{tcolorbox}

	\begin{tcolorbox}[colback=gray!5!white,colframe=gray!75!black]
		\textbf{\large{Q12}} 
	\end{tcolorbox}

	\begin{tcolorbox}[colback=gray!5!white,colframe=gray!75!black]
		\textbf{\large{Q13}} 
	\end{tcolorbox}
	
\end{document}