\documentclass[french]{article}
\usepackage[T1]{fontenc}
\usepackage[utf8]{inputenc}
\usepackage{lmodern}
\usepackage[a4paper]{geometry}
\usepackage{babel}
\usepackage{amsmath}
\usepackage{amsfonts}
\usepackage{graphicx}  % This package allows the importing of images
\usepackage{tcolorbox}
\usepackage{color}
\usepackage{breqn}

\definecolor{dkgreen}{rgb}{0,0.6,0}
\definecolor{gray}{rgb}{0.5,0.5,0.5}
\definecolor{mauve}{rgb}{0.58,0,0.82}

\begin{document}
	
\title{PC 6: Convergence \& Loi des grands nombres}
\date{Dernière modification \today}

\maketitle

\begin{tcolorbox}[colback=red!5!white,colframe=red!75!black]
	\textbf{\large{Rappels}}
	\begin{itemize}
		\item Convergence en probabilité
		\[X_n \xrightarrow{\mathbb{P}} X \iff \forall \epsilon > 0 \lim_{n \to +\infty} \mathbb{P}(|X_n - X| > \epsilon) = 0\]
		\item Convergence em moyenne
		\[X_n \xrightarrow{L^1} X \iff  \lim_{n \to +\infty} \mathbb{E}(|X_n - X|) = 0\]
		\item Convergence presque-sûrement
		\[X_n \xrightarrow{\text{p.s}} X \iff  \mathbb{P}\left(\{\omega : \lim_{n \to +\infty} X_n(\omega) = X(\omega)\}\right) = 1\]
		\item Théorème de convergence dominée
		\[\text{Si } X_n \xrightarrow{\text{p.s}} X \text{ et } \exists Z \text{ v.a réele tel que } \forall n, |X_n| \leq |Z| \text{ p.s. Alors, } X_n \xrightarrow{L^1} X\]
		\item Loi des grands nombres
		Soit $(X_n)_{n \geq 0}$ i.i.d $\in L^1$.
		\[\frac{1}{n}\sum_{i=1}^{n} X_i \xrightarrow{\text{p.s}} \mathbb{E}(X_1)\]
		\item Lemme de Borel-Cantelli
		\begin{itemize}
			\item Si la série $\sum_n \mathbb{P}(A_n) < +\infty$, alors $\mathbb{P}(\lim \sup A_n) = 0$. C'est à dire, p.s il y a un nombre fini de $A_n$ qui sont réalisés.
			\item Si de plus la suite $(A_n)_n$ est indépendante, alors $\sum_{n} \mathbb{P}(A_n) = +\infty \implies \mathbb{P}(\lim \sup A_n) = 1$. C'est à dire, p.s une infinité de $A_n$ sont réalisés.
		\end{itemize}
	\end{itemize}
\end{tcolorbox}

\newpage

\begin{tcolorbox}[colback=gray!5!white,colframe=gray!75!black]
	\textbf{\large{Q1}} Soit $(X_n)_{n \geq 1}$ une suite de variables aléatoires de loi Bernoulli, $X_n \sim \mathcal{B}(p_n)$ avec $p_n \to 0$. 
	\begin{enumerate}
		\item Montrer que $X_n \xrightarrow{\mathbb{P}} 0$ et $X_n \xrightarrow{\mathbb{L}^1} 0$.
		\item On suppose que $\sum_{n} p_n < \infty$. Montrer que $X_n \xrightarrow{p.s} 0$.
		\item On suppose que $X_n \xrightarrow{p.s} 0$. Montrer qu'on n'a pas nécessairement $\sum_{n}p_n < \infty$.
		\item On suppose maintenant que $\sum_{n}p_n = \infty$ et que les variables $(X_n)_{n \geq 1}$ sont indépendantes. Montrer que $X_n \not\to X$ p.s.
	\end{enumerate}
\end{tcolorbox}

\begin{enumerate}
	
\item
\[\forall \epsilon>0 \quad \mathbb{P}(|X_n - 0| > \epsilon) = \mathbb{P}(X_n = 1) = p_n \to 0\]
donc, $X_n \xrightarrow{\mathbb{P}} 0$.

\[\mathbb{E}(|X_n - 0|) = \mathbb{E}(X_n) = p_n \to 0\]
donc, $X_n \xrightarrow{L^1} 0$.

\item
Par le lemme de Borel-Cantelli $\sum_n p_n < +\infty \implies \mathbb{P}(\lim \sup \{X_n = 1\})$. Donc, il y a un nombre fini de $X_i = 1 \implies X_n \xrightarrow{p.s} 0$.

\[\mathbb{P}\left(\{\omega : \lim_{n \to +\infty} X_n(\omega) \neq X(\omega)\}\right) = \mathbb{P}(\lim \sup \{X_n = 1\}) = 0\]

\item
Soit $U$ une v.a réele de loi uniforme $U \sim \mathcal{U}[0,1]$.
Prenons $X_n = \mathfrak{1}_{[0, \frac{1}{n}]}(U)$. On note $X_n \sim \mathbb{B}(\frac{1}{n})$.
\[\sum_{n} p_n = \sum_n \frac{1}{n} = +\infty\]
$X_n(\omega) = \mathfrak{1}_{[0, \frac{1}{n}]}(U(\omega)) = 0 \forall n \geq n_0 \implies X_n \xrightarrow{p.s} 0$.

\item
Par le lemme de Borel-Cantelli $\sum_{n} p_n = +\infty$ et la suite $(X_n)_n$ est indépendante, alors $\mathbb{P}(\lim \sup A_n) = 1$. Donc, on a une infinitude de $X_i =1 \implies X_n \xrightarrow{p.s} 1$.

\end{enumerate}

\newpage

\begin{tcolorbox}[colback=gray!5!white,colframe=gray!75!black]
	\textbf{\large{Q4}}  Montrer qu'une suite $(X_n)_{n \in \mathbb{N}}$ de variables aléatoires converge presque sûrement vers une variable aléatoire $X$ si et seulement si $M_n = \sup_{k \geq n} |X_k - X|$ converge vers $0$ en probabilité.
\end{tcolorbox}

Supposons que $X_n \xrightarrow{p.s} X \implies M_n \xrightarrow{p.s} 0 \implies M_n \xrightarrow{\mathbb{P}} 0$.

Réciproquement, supposons que $M_n \xrightarrow{\mathbb{P}} 0$.
\begin{align*}
	\iff \forall \epsilon > 0 \lim\limits_{n \to +\infty} \mathbb{P}(|M_n - 0| > \epsilon) &= 0 \\
	&= \mathbb{P}(M_n > \epsilon)\\
	&= \mathbb{P}(\sup_{k \geq n} |X_k - X| > \epsilon)\\
	&= \mathbb{P}(\cup_{k \geq n} \{|X_k - X | > \epsilon\})\\
	\iff \forall \epsilon > 0 \quad \mathbb{P}(\cap_{n \geq 1} \cup_{k \geq n} \{|X_k - X| > \epsilon\}) = 0
\end{align*}

En particulier, soit $B = \cup_{m \in \mathbb{N}} \cap_{n \geq 1} \cup_{k \geq n} \{|X_k - X| > \frac{1}{m}\}$ on a

\[\mathbb{P}\left(\cup_{m \in \mathbb{N}} \cap_{n \geq 1} \cup_{k \geq n} \{|X_k - X| > \frac{1}{m}\}\right) = 0.\]

$\mathbb{P}(\lim X_n \neq X) \leq \mathbb{P}(B) = 0 \implies X_n \xrightarrow{\text{p.s}} X$.

\newpage

\begin{tcolorbox}[colback=gray!5!white,colframe=gray!75!black]
	\textbf{\large{Q6}} Soit $g: \mathbb{R} \to \mathbb{R}$ une fonction continue bornée et $\lambda > 0$. Déterminer
	\begin{enumerate}
		\item
		\[\lim_{n \to \infty} \int_{[0,1]^n} g\left(\frac{x_1+...+x_n}{n}\right)dx_1...dx_n\]
		\item
		\[\lim_{n \to \infty} \sum_{k=0}^{+\infty} e^{-\lambda n} \frac{(\lambda n)^k}{k!}g\left(\frac{k}{n}\right)\]
	\end{enumerate}
\end{tcolorbox}

\begin{enumerate}
	\item Soit $X_i \sim \mathcal{U}[0,1]$ indépendantes.
	\begin{align}
		\lim_{n \to \infty} \int_{[0,1]^n} g\left(\frac{x_1+...+x_n}{n}\right)dx_1...dx_n &= \int_{[0,1]^n} g(\frac{1}{n} \sum_{i=1}^{n}X_i) dx_1...dx_n\\
		&= \mathbb{E}(g(\bar{X_n}))\\
		&\xrightarrow{\text{p.s}} \mathbb{E}(g(\mathbb{E}(X_i)))  \text{ par la loi des grands nombres. }\\
		&= g\left(\frac{1}{2}\right)
	\end{align}
	
	\item Soit $Y_n \sim \mathcal{P}(\lambda k)$.
	\begin{align}
		\lim_{n \to \infty} \sum_{k=0}^{+\infty} e^{-\lambda n} \frac{(\lambda n)^k}{k!}g\left(\frac{k}{n}\right) &= \mathbb{E}\left(g\left(\frac{Y_n}{n}\right)\right)\\
		&= \mathbb{E}(g(\bar{Z_n})) \text{ où $Z_i \sim \mathcal{P}(\lambda)$ indépendantes }\\
		&\xrightarrow{\text{p.s}} g(\lambda) \text{ par la loi des grands nombres. }
	\end{align}
	
\end{enumerate}


\newpage

\begin{tcolorbox}[colback=gray!5!white,colframe=gray!75!black]
	\textbf{\large{Q9}}  On considère une particule se déplaçant sur l'axe réel, et on note $X_n$ sa position à l'instant $n \in \mathbb{N}$. On suppose que $X_0$ est une v.a réelle et que la position de la particule évolue de la manière suivante
	\[X_{n+1} = X_{n} + \epsilon_{n+1} \quad \forall n \in \mathbb{N},\]
	où $(\epsilon_n)_{n \geq 1}$ est une suite de v.a.r i.i.d intégrables et de moyenne $m \neq 0$.
	Montrer que $\lim_{n\to\infty}|X_n| = +\infty$ p.s.
\end{tcolorbox}

Par récurrence $X_n = X_0 + \sum_{i=1}^{n} \epsilon_n$.

Par la loi des grands nombres,
\[\iff \frac{1}{n}X_n = \frac{1}{n}X_0 + \frac{1}{n}\sum_{i=1}^{n} \xrightarrow{\text{p.s}} \mathbb{\epsilon_1} = m \neq 0.\]

Donc $X \approx nm$ pour $n$ grand $\implies |X_n| \xrightarrow{\text{p.s}} +\infty$.

\end{document}