\documentclass[french]{article}
\usepackage[T1]{fontenc}
\usepackage[utf8]{inputenc}
\usepackage{lmodern}
\usepackage[a4paper]{geometry}
\usepackage{babel}
\usepackage{amsmath}
\usepackage{amsfonts}
\usepackage{tcolorbox}
\usepackage{color}
\usepackage{breqn}

\begin{document}
	\title{DM1: Oscillations d'atomes piégés dans un potentiel parabolique}
	\date{Dernière modification \today}
	
	\maketitle
	
	\section{Mesure par vol libre de la densité de probabilité de l'impulsion}
	
	\begin{tcolorbox}[colback=gray!5!white,colframe=gray!75!black]
		\textbf{\large{Q1}} Rappeler sans démonstration l'expression des niveaux d'énergie du système.
	\end{tcolorbox}

	Les niveaux d'énergie de l'oscillateur harmonique à une dimension sont
	
	\[E_n = \left(n + \frac{1}{2}\right)\hbar \omega\]

	\begin{tcolorbox}[colback=gray!5!white,colframe=gray!75!black]
		\textbf{\large{Q2}} Compte tenu de la valeur numérique de $a_0$, pensez-vous qu'il soit possible de résoudre l'extension spatiale de l'etat fondamental à l'aide d'un microscope optique utilisant la lumière visible ?
	\end{tcolorbox}

	\begin{align}
		a_0 &= \sqrt{\frac{\hbar}{m \omega}}\\
		&= \sqrt{\frac{1,055 \cdot 10^{-34} \text{Js}}{(2,2 \cdot 10^{-25} \text{kg}) \cdot (5,718 \cdot 10^5 \text{rad/s})}}\\
		&= 2,9 \cdot 10^{-8} \text{m}
	\end{align}
	
	Sachant que la résolution d'un microscope optique est $0,2 \text{\mu m} \implies$ ce n'est pas possible de résoudre l'extension spatiale de l'état fondamental $a_0 \approx 0,03 \text{\mu m}$.
	
	\begin{tcolorbox}[colback=gray!5!white,colframe=gray!75!black]
		\textbf{\large{Q3}} Au lieu de mesurer la densité de probabilité de la position, on choisit de mesurer la densité de probabilité de l'impulsion, $|\phi (p, t_0)|^2$, à un instant $t_0$ donné. Pour cela, on éteint brusquement le laser de piégeage à l'instant $t_0$, de sorte que les atomes se comportent comme un paquet d'ondes libre pour $t > t_0$. On rappelle que dans ce cas, pour $t - t_0$ suffisamment grand, on a 
		\[|\psi (x,t)|^2 \propto \left| \phi \left(\frac{mx}{t - t_0}, t_0\right) \right|^2\]
		(méthode dite du \textit{temps de vol} ou du \textit{vol libre}). L'image obtenue reflète ainsi la densité de probabilité de l'impulsion $|\phi (p, t_0)|^2$ à l'instant $t_0$ où le laser de piégeage a été coupé.
		Commenter la figure 1(a), obtenue de cette manière lorque $|\psi (t_0) \rangle = |0\rangle$, et estimer un \textit{ordre de grandeur} du temps de vol choisit, $T_v = t - t_0$.
	\end{tcolorbox}

	On note que la distribution est une gaussienne comme attendu pour l'état fondamental $|0\rangle$ avec écart-type $\sigma \approx 100 \text{\mu m}$.
	
	Pour estimer le temps de vol, on utilise la relation
		
	\[\Delta v_0 = \frac{\Delta x(T_v)}{T_v}\]
	
	On note que
	
	\begin{align}
		\begin{cases}
		\Delta v_0 &= \frac{\Delta p_0}{m} = \frac{1}{m} \cdot \frac{\hbar}{a_0} = 1,65 \cdot 10^{-2} \text{m/s}\\
		\Delta x(T_v) &\approx 10^{-4} \text{m}
		\end{cases}
	\end{align}
	
	Donc, $T_v \approx 6 \text{ms}$ dans l'ordre de $10^{-3}$s.
	
	
	\begin{tcolorbox}[colback=gray!5!white,colframe=gray!75!black]
		\textbf{\large{Q4}}On considère une fonction d'onde $\psi(x)$ ainsi que sa transformée de Fourier $\phi(p)$. Ecrire l'expression de $\hat{a}^\dag \psi(x)$ sous forme d'un opérateur différentiel, puis montrer que 
		\[\hat{a}^\dag \phi(p) = \frac{-i}{\sqrt{2}} \left(\frac{a_0}{\hbar}p - \frac{\hbar}{a_0} \frac{d}{dp}\right)\phi(p).\]
	\end{tcolorbox}

	On veut calculer $\hat{a}^\dag$.
	
	\begin{align}
		\hat{a}^\dag &= \frac{1}{\sqrt{2}} \left(\frac{\hat{x}}{a_0} + i\frac{a_0}{\hbar}\hat{p}\right)^\dag\\ 
		&= \frac{1}{\sqrt{2}} \left(\frac{\hat{x}^\dag}{a_0} - i\frac{a_0}{\hbar}\hat{p}^\dag\right)\\
		&= \frac{1}{\sqrt{2}} \left(\frac{\hat{x}}{a_0} - i\frac{a_0}{\hbar}\hat{p}\right)
	\end{align}
	
	Donc, 
	\begin{align}
		\hat{a}^\dag \psi(x) &= \frac{1}{\sqrt{2}} \left(\frac{\hat{x}}{a_0} - i\frac{a_0}{\hbar}\hat{p}\right) \psi(x)\\
		&= \frac{1}{\sqrt{2}} \left(\frac{x}{a_0} - a_0\frac{d}{dx}\right) \psi(x).
	\end{align}
	
	Pour calculer $\hat{a}^\dag \phi(p)$, on applique la transformée de Fourier
	
	\begin{align}
		\begin{cases}
			x \psi(x) &\mapsto i\hbar \frac{d}{dp}\phi(p)\\
			\frac{d}{dx} \psi(x) &\mapsto \frac{ip}{\hbar} \phi(p)
		\end{cases}
	\end{align}
	
	Donc,
	\begin{align}
		\hat{a}^\dag \phi(p) &= \frac{1}{\sqrt{2}} \left(\frac{i\hbar}{a_0}\frac{d}{dp} - a_0\frac{ip}{\hbar}\right) \phi(p)\\
		&= \frac{-i}{\sqrt{2}} \left(-\frac{\hbar}{a_0}\frac{d}{dp} + \frac{a_0p}{\hbar}\right) \phi(p)
	\end{align}

	\begin{tcolorbox}[colback=gray!5!white,colframe=gray!75!black]
		\textbf{\large{Q5}}En déduire que l'on peut écrire $\phi_1(p) = \xi p \phi_0(p)$ où $\xi$ est un nombre complexe que l'on déterminera. 
	\end{tcolorbox}

	On utilise la relation $\hat{a}^\dag |n\rangle = \sqrt{n+1} |n+1\rangle$ avec $n = 0$.
	
	\begin{align}
		\phi_1(p) &= \hat{a}^\dag \phi_0(p)\\
		&= \frac{-i}{\sqrt{2}} \left(\frac{a_0p}{\hbar} -\frac{\hbar}{a_0}\frac{d}{dp}\right) \phi_0(p)\\
		&= \left(\frac{-i}{\sqrt{2}}\right)\left(\frac{a_0p}{\hbar} - \frac{\hbar}{a_0}\left(-\frac{a_0^2p}{\hbar^2}\right)\right)\left(\frac{a_0^2}{\pi \hbar^2}\right)^{\frac{1}{4}}\text{exp}\left(-\frac{a_0^2p^2}{2\hbar^2}\right)\\
		&=\left(\frac{-i}{\sqrt{2}}\right) \frac{2a_0p}{\hbar}\left(\frac{a_0^2}{\pi \hbar^2}\right)^{\frac{1}{4}}\text{exp}\left(-\frac{a_0^2p^2}{2\hbar^2}\right)\\
		&= \left(\frac{-i}{\sqrt{2}}\right) \frac{2a_0p}{\hbar}\phi_0(p)\\
		&= \left(-\frac{\sqrt{2}ia_0}{\hbar}\right)p\phi_0(p)
	\end{align}
	
	Donc, $\phi_1(p) = \xi p \phi_0(p)$ avec $\xi = -\frac{\sqrt{2}ia_0}{\hbar}$


	\begin{tcolorbox}[colback=gray!5!white,colframe=gray!75!black]
		\textbf{\large{Q6}} Commenter la Figure 1(b).
	\end{tcolorbox}

	La fonction d'onde $\psi_1(x)$ est proportionnel à la fonction d'Hermite $H_1(\frac{x}{a_0}) \implies $ la densité $|\psi_1(t)|^2$ est proportionnel à $H_1(\frac{x}{a_0})^2$. On observe dans la figure 1(b) la forme attendu avec $2$ nuages distinctes.

	\section{Préparation du système dans le premier état excité}

	\begin{tcolorbox}[colback=gray!5!white,colframe=gray!75!black]
		\textbf{\large{Q1}} Déterminer les états propres de $\hat{H_1} = \frac{\hbar \Omega}{2} \left(|1\rangle \langle 0| + |0\rangle \langle 1| \right)$  ainsi que les valeurs propres correspondantes.
	\end{tcolorbox}

	On veut résoudre $\langle \hat{H_1} | \psi \rangle = \lambda \psi(x)$
	
	\begin{align}
	\langle \hat{H_1} | \psi \rangle &= \frac{\hbar \Omega}{2} \left( \langle 0|\psi \rangle |1\rangle + \langle 1|\psi \rangle |0\rangle\right)\\
	\end{align}
	
	Donc, \[\frac{\hbar \Omega}{2} \left( \langle 0|\psi \rangle |1\rangle + \langle 1|\psi \rangle |0\rangle\right) = \lambda \psi(x)\]
	
	On note que $\psi(x)$ doit être une combinaison lineaire de $|0\rangle$ et $|1\rangle$. Donc, \[\psi(x) = \lambda_0 \cdot \psi_0(x) + \lambda_1 \cdot \psi_1(x)\]
	
	Alors,
	
	\begin{align}
		\begin{cases}
		\lambda \lambda_0 \psi_0(x) &= \frac{\hbar \Omega}{2} \langle 1|\psi \rangle |0\rangle\\
		\lambda \lambda_1 \psi_1(x) &= \frac{\hbar \Omega}{2} \langle 0|\psi \rangle |1\rangle
		\end{cases}
	\end{align}
	
	On note que $ \langle 0|\psi \rangle = \lambda_0$ et $ \langle 1|\psi \rangle = \lambda_1$ parce que $|0\rangle$ et $|1\rangle$ sont orthogonaux. Donc,
	
	\begin{align}
	\begin{cases}
	\lambda \lambda_0 &= \frac{\hbar \Omega}{2} \lambda_1\\
	\lambda \lambda_1 &= \frac{\hbar \Omega}{2} \lambda_0
	\end{cases}
	\end{align}
	
	En divisant les deux équation on a $\lambda_0^2 = \lambda_1^2$.
	
	\begin{align}
		\begin{cases}
		\lambda_0 = \lambda_1 \iff \lambda = \frac{\hbar \Omega}{2}\\
		\lambda_0 = -\lambda_1 \iff  \lambda = \frac{\hbar \Omega}{2}
		\end{cases}
	\end{align}
	
	Donc, les états propres sont 
	\begin{align}
		\psi(x) &= c(\psi_0(x) + \psi_1(x)) \text{où $c \in \mathbb{C}$ avec valeur propre } \frac{\hbar \Omega}{2}\\
		\psi(x) &= c(\psi_0(x) - \psi_1(x)) \text{où $c \in \mathbb{C}$ avec valeur propre } -\frac{\hbar \Omega}{2}
	\end{align}

	\begin{tcolorbox}[colback=gray!5!white,colframe=gray!75!black]
		\textbf{\large{Q2}} Décomposer l'état $|\psi(0)\rangle$ dans la base propre obtenue à la question précédente, puis en déduire pour $t > 0$ l'expression de $|\psi(t)\rangle$ dans cette même base.
	\end{tcolorbox}

	On utilise la base orthonormale $\left\{ \frac{|0\rangle + |1\rangle}{\sqrt{2}} ,\frac{|0\rangle - |1\rangle}{\sqrt{2}}  \right\}$.
	
	\[|\psi(0)\rangle = |0\rangle = \frac{1}{\sqrt{2}}\left(\frac{|0\rangle + |1\rangle}{\sqrt{2}}\right) + \frac{1}{\sqrt{2}}\left(\frac{|0\rangle - |1\rangle}{\sqrt{2}}\right) \]
	
	On utilise $\psi(x,t) = \psi(x,t=0) \cdot e^{-\frac{i\lambda}{\hbar}t}$ où $\lambda$ est le valeur propre correspondant.
	
	\[|\psi(t)\rangle = \frac{e^{-i \frac{\Omega}{2} t}}{\sqrt{2}}\left(\frac{|0\rangle + |1\rangle}{\sqrt{2}}\right) + \frac{e^{+i \frac{\Omega}{2} t}}{\sqrt{2}}\left(\frac{|0\rangle - |1\rangle}{\sqrt{2}}\right)\]

	\begin{tcolorbox}[colback=gray!5!white,colframe=gray!75!black]
		\textbf{\large{Q3}} Ecrire  $|\psi(t)\rangle$ dans la base propre de $\hat{H_0}$.
	\end{tcolorbox}

	Dans la base $\{ |0\rangle, |1\rangle\}$
	
	\begin{align}
	|\psi(t)\rangle &= \left(\frac{e^{-i \frac{\Omega}{2} t} +e^{+i \frac{\Omega}{2} t} }{2}\right) |0\rangle + \left(\frac{e^{-i \frac{\Omega}{2} t} - e^{+i \frac{\Omega}{2} t} }{2}\right) |1\rangle\\
	&= \cos\left(\frac{\Omega t}{2}\right)|0\rangle - i \sin\left(\frac{\Omega t}{2}\right)|1\rangle
	\end{align}

	\begin{tcolorbox}[colback=gray!5!white,colframe=gray!75!black]
		\textbf{\large{Q4}} Calculer la probabilité $\mathcal{P}(t)$ qu'une mesure de $H_0$ effectué à l'instant $t$ donne le résultat $3\hbar \omega /2$, puis montrer que cette fonction est une fonction périodique dont on déterminera la période $T$. 
	\end{tcolorbox}

	La mesure donne le résultat $\frac{3\hbar \omega}{2} \iff$ le système est dans l'état $|1\rangle$.
	
	\begin{align}
		\mathcal{P}(t) &= \left|-i \sin\left(\frac{\Omega t}{2}\right)\right|^2\\
		&= \sin\left(\frac{\Omega t}{2}\right)^2\\
		&= \frac{1 - cos\left(\Omega t\right)}{2}
	\end{align}
	
	Donc, $\mathcal{P}(t)$ est périodique avec période $T = \frac{2\pi}{\Omega}$

	\begin{tcolorbox}[colback=gray!5!white,colframe=gray!75!black]
		\textbf{\large{Q5}} A quel instant faut-il interrompre l'application du second laser pour placer le système dans l'état $|1\rangle$ (à une phase près) ? 
	\end{tcolorbox}

	\[\mathcal{P}(t) = 1 \iff \cos(\Omega t) = -1 \iff t = \frac{\pi}{\Omega} = \frac{T}{2}\]

	\section{Préparation d'un état non stationnaire}

	\begin{tcolorbox}[colback=gray!5!white,colframe=gray!75!black]
		\textbf{\large{Q1}} Ecrire dans la base propre de $\hat{H_0}$ l'état du système à l'instant $t = T/4$, sachant que $|\psi(0)\rangle = |0\rangle$.
	\end{tcolorbox}

	On note $t = \frac{T}{4} = \frac{\pi}{2\Omega} \implies \frac{\Omega t}{2} = \frac{\pi}{4}$

	\begin{align}
	|\psi(\frac{T}{4})\rangle &= \cos\left(\frac{\pi}{4}\right)|0\rangle - i \sin\left(\frac{\pi}{4}\right)|1\rangle\\
	&= \frac{|0\rangle}{\sqrt{2}} - i \frac{|1\rangle}{\sqrt{2}}
	\end{align}

	\begin{tcolorbox}[colback=gray!5!white,colframe=gray!75!black]
		\textbf{\large{Q2}} En déduire l'expression de $|\psi(t = T/4 + \tau)\rangle$ pour $\tau > 0$.
	\end{tcolorbox}

	On note que $|0\rangle$ et $|1\rangle$ sont des états propres du hamiltonien $\hat{H_0}$ avec les valeurs propres $\frac{\hbar \omega}{2}$ et $\frac{3 \hbar \omega}{2}$. Donc,

	\[|\psi(t = T/4 + \tau)\rangle = e^{-i\frac{\omega}{2}\tau}\frac{|0\rangle}{\sqrt{2}} - i e^{-i\frac{3\omega}{2}\tau}\frac{|1\rangle}{\sqrt{2}}\]

	\begin{tcolorbox}[colback=gray!5!white,colframe=gray!75!black]
		\textbf{\large{Q3}} Exprimer la densité de probabilité $|\phi(p,t)|^2$ en fonction de $|\phi_0(p)|^2$.
	\end{tcolorbox}

	On applique la transformée de Fourier
	
	\[\phi(t = \frac{T}{4}+\tau) = e^{-i\frac{\omega}{2}\tau}\frac{\phi_0(p)}{\sqrt{2}} - i e^{-i\frac{3\omega}{2}\tau}\frac{\phi_1(p)}{\sqrt{2}}\]
	
	Donc,
	\begin{align}
		|\phi(p, t)|^2 &= \frac{1}{2}\left|\phi_0(p) - ie^{-i\omega\tau}\phi_1(p)\right|^2\\
		&= \frac{1}{2}\left|1 - i\xi pe^{-i\omega\tau}\right|^2 |\phi_0(p)|^2
	\end{align}
	
	Avec $\xi = -\frac{\sqrt{2}ia_0}{\hbar}$, on note que
	
	\begin{align}
		\left|1 - i\xi pe^{-i\omega\tau}\right|^2 &= \left|  1 - \frac{\sqrt{2}a_0}{\hbar}pe^{-i\omega\tau} \right|^2\\
		&= \left|  1 - \frac{\sqrt{2}a_0}{\hbar}p\cos(\omega \tau) + i\frac{\sqrt{2}a_0}{\hbar}p\sin(\omega \tau) \right|^2\\
		&= 1 - \frac{2\sqrt{2}a_0}{\hbar}p\cos(\omega \tau) + \frac{2a_0^2}{\hbar^2}p^2\cos^2(\omega \tau) +  \frac{2a_0^2}{\hbar^2}p^2\sin^2(\omega \tau)\\
		&= 1 + \frac{2a_0^2}{\hbar^2}p^2 - \frac{2\sqrt{2}a_0}{\hbar}p\cos(\omega \tau)
	\end{align}
	
	Donc, 
	\begin{align}
	|\phi(p, t)|^2 &= \left(\frac{1}{2} + \frac{a_0^2}{\hbar^2}p^2 - \frac{\sqrt{2}a_0}{\hbar}p\cos(\omega \tau)\right) |\phi_0(p)|^2
	\end{align}

	\begin{tcolorbox}[colback=gray!5!white,colframe=gray!75!black]
		\textbf{\large{Q4}} La figure 2 répresente la densité de probabilité obtenue par la méthode du vol libre pour différentes valeurs du temps d'évolution $\tau = q \tau_0$, où $q$ est un entier. Justifier qualitativement la forme de ces courbes puis déterminer la valeur de $\tau_0$.
	\end{tcolorbox}

	On sait que $phi(p, t)|^2 = \frac{1}{2}\left|\phi_0(p) - ie^{-i\omega\tau}\phi_1(p)\right|^2$.
	
	Donc, conforme $\tau$ varie, les coefficients de la superposition entre $|0\rangle$ et $|1\rangle$ changent périodiquement. C'est pourquoi on voit la courbe c'est se déplacer à droite et après à gauche périodiquement.
	
	On note que le période est $7\tau_0$. Donc,
	
	\[\frac{2\pi}{\omega} = 7\tau_0 \iff \tau_0 = 1,57 \cdot 10^{-6}s\] 
	

\end{document}